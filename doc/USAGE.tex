\documentclass{amsart}

\newcommand{\br}{{\mathbf r}}
\newcommand{\e}{{\mathbf e}}
\newcommand{\brho}{{\boldsymbol \rho}}
\newcommand{\bnabla}{{\boldsymbol \nabla}}

\title{Using the Spherical Software}

\begin{document}

\maketitle

\noindent
The {\tt spherical} software is invoked by the command
\begin{center}
{\tt spherical -p parameter-file.txt}
\end{center}
where {\tt parameter-file.txt} is a text file that sets various parameters, described below.  The command can be invoked with two other command line options, for example:
\begin{center}
{\tt spherical -p parameter-file.txt -o output-file -v10}
\end{center}
where {\tt output-file} is the output file, which will default to {\tt s.out} if not specified, and {\tt -v} denotes verbose output, and if followed by a number $n$, will write out to the console every $n$th time it writes data to the output file.

The program {\tt spherical} uses spherical harmonics to solve Jeffery's equation \cite{jeffery} for $\psi=\psi(\br)$:
$$ \frac\partial{\partial t}\psi = \bnabla\cdot\bigl(-\dot\br \psi + \bnabla(D_r \psi)\bigr) ,$$
where
$$ \dot\br = \tfrac12(\mathbf w \times\br) + \lambda \dot{\tilde\br} ,\quad \dot{\tilde\br} = \tfrac12(\Gamma \cdot\br - \Gamma:\br\br\br) , $$
where $\Gamma$ is the rate of strain tensor, and $\mathbf w$ is the vorticity.  Note that $\bnabla$ here denotes the gradient projected tangentially to the sphere, namely $\bnabla f = (I-\br\br)\cdot(\tfrac{\partial f}{\partial x},\tfrac{\partial f}{\partial y},\tfrac{\partial f}{\partial z})$.

\section*{Basic Parameters}

\noindent
The parameter file consists of a number of lines, each of the form
\begin{center}
{\tt parameter=value}
\end{center}
The order in which they are written in the parameter file is unimportant.

\begin{itemize}
\item The number of threads used by the program is set by {\tt nr\_threads}.  This can be overridden by the environment variable {\tt NR\_THREADS} (using,for example, the {\tt SET} command in the windows command shell, or the {\tt export} or {\tt env} commands in the unix {\tt sh} or {\tt bash} shells.)
\item The parameter {\tt max\_order} sets the order of spherical harmonics used.
\item The start time and stop time are set by parameters {\tt tstart} and {\tt tend}.
\item The rate of strain tensor is set by parameters {\tt gamma11}, {\tt gamma12}, {\tt gamma13}, {\tt gamma22}, {\tt gamma23}, {\tt gamma33}.
\item The vorticity is set by parameters {\tt w1}, {\tt w2}, {\tt w3}.
\item The initial state is $\psi = 1/4\pi$.  Later versions of the software may allow this to be different.
\item By default the program will solve the differential equations using the order two-three Runge-Kutta-Fehlberg method.
\item The parameter set as {\tt ode\_rkf\_45=1} will cause the program will use the order four-five Runge-Kutta-Fehlberg method.
\item The parameter set as {\tt ode\_adams\_bash\_2=1} will cause the program will use the order two Adams-Bashforth multistep method.
\item The parameter set as {\tt ode\_adams\_bash\_4=1} will cause the program will use the order four Adams-Bashforth multistep method.
\item The parameter set as {\tt ode\_rk\_4=1} will cause the program will use the order four Runge-Kutta method.
\item For fixed timestep methods (Runge-Kutta or Adams-Bashforth), the step size of time is set by the parameter {\tt h}.  For adaptive methods, this parameter sets the initial step size of time.  This defaults to $10^{-3}$ if not set.
\item For the adaptive Runge-Kutta-Fehlberg methods, the parameter {\tt tol} sets the desired error, which is computed as the maximum of the absolute values of the spherical harmonic coefficients.  This defaults to $10^{-3}$ if it is not set.
\item Data will be printed to the output file every {\tt print\_every} time steps.  This defaults to $1$ if it is not set.
\end{itemize}

\section*{Output}

\noindent
The program, by default, will write to the output file a series of lines containing
\begin{center}$t$ $a_{11}$ $a_{22}$ $a_{33}$ $a_{12}$\end{center}
This same information is output to the console if the command line option {\tt -v} with a number is specified.  Here $a_{ij}$ are the coefficients of the second moments tensor.  Similarly $a_{ijkl}$ are the coefficients of the fourth moments tensor, and $a_{ijklmn}$ are the coefficients of the sixth moments tensor.
\begin{itemize}
\item The parameter set as {\tt print\_aij=1} will cause the lines written to the output file to change to
\begin{center}$t$ $a_{11}$ $a_{12}$ $a_{13}$ $a_{22}$ $a_{23}$ $a_{33}$\end{center}
(but the output to the console will remain as before).
\item The parameter set as {\tt print\_aijkl=1} will add to the end of each line of output the values\\
$a_{1111}$ $a_{1112}$ $a_{1113}$ $a_{1122}$ $a_{1123}$ $a_{1133}$ $a_{1222}$ $a_{1223}$ $a_{1233}$ $a_{1333}$ $a_{2222}$ $a_{2223}$ $a_{2233}$ $a_{2333}$ $a_{3333}$
\item The parameter set as {\tt print\_aijklmn=1} will add to the end of each line of output the values\\
$a_{111111}$ $a_{111112}$ $a_{111113}$ $a_{111122}$ $a_{111123}$ $a_{111133}$ $a_{111222}$ $a_{111223}$ $a_{111233}$ $a_{111333}$ $a_{112222}$ $a_{112223}$ $a_{112233}$ $a_{112333}$ $a_{113333}$ $a_{122222}$ $a_{122223}$ $a_{122233}$ $a_{122333}$ $a_{123333}$ $a_{133333}$ $a_{222222}$ $a_{222223}$ $a_{222233}$ $a_{222333}$ $a_{223333}$ $a_{233333}$ $a_{333333}$
\item The parameters {\tt print\_daij}, {\tt print\_daijkl}, and {\tt print\_daijklmn} set to 1 will print out the derivatives with respect to time of, respectively, the second, fourth, and sixth order moment tensors.
\end{itemize}

\section*{Folgar-Tucker model}

\noindent
The program will default to solving Jeffery's equation with the Folgar-Tucker diffusion term \cite{folgar}.  This sets the diffusion term to $ D_r = C_I \gamma $.
Here, and elsewhere, $\gamma = \left(\frac12\Gamma:\Gamma\right)^{1/2}$.
\begin{itemize}
\item The parameter {\tt lambda} is the Jeffery's parameter $\lambda$.
\item The parameter {\tt CI} sets the parameter $C_I$.
\end{itemize}

\section*{Modified Koch model}

\noindent
This provides an anisotropic diffusion model:
\begin{align*}
\frac\partial{\partial t}\psi &= - \bnabla\cdot(\dot\br \psi) + \bnabla\cdot(I-\br\br)\cdot D_r\cdot\bnabla\psi \\
&\qquad - \mathbf L\cdot(I-\br\br)\cdot E_r\cdot\mathbf L \psi \\
& = - \bnabla\cdot(\dot\br \psi) + (\bnabla-2\br)\cdot D_r\cdot\bnabla\psi - \mathbf L \cdot E_r \cdot \mathbf L \psi .
\end{align*}
Here $(I-\br\br)$ is the matrix of the projection onto the tangent space of the sphere, and $\mathbf L = -i \br \times \bnabla$ is the so called angular momentum operator.  (Note that the Laplacian operator can be expressed two different ways as $\bnabla\cdot\bnabla = -\mathbf L \cdot \mathbf L$.)

Koch \cite{koch} defines $D_r$ as the matrix
$$ D_r = C_1 \gamma^{-1} (\Gamma: \mathbb A:\Gamma) I + C_2 \gamma^{-1} \Gamma: \mathcal A:\Gamma .$$
Here $A$, $\mathbb A$, and $\mathcal A$ are the tensors of $2$nd, $4$th and $6$th moments, respectively, of $\psi$.  This software has an additional modification in the form of $E_r$:
$$ E_r = C_3 \gamma^{-1} \Gamma: \mathcal A:\Gamma .$$
\begin{itemize}
\item The parameter set as {\tt do\_koch=1} will cause the modified Koch diffusion model to be used.
\item The parameter {\tt lambda} is the Jeffery's parameter $\lambda$.
\item The parameters {\tt C1}, {\tt C2} and {\tt C3} are the modified Koch parameters $C_1$, $C_2$ and $C_3$.
\end{itemize}

\section*{Directional diffusion model}

\noindent
This gives values for $D_r=D_r(\br)$ that depend upon $\br$, in the diffusion term of Jeffery's equation.  The directional diffusion model is described in the Ph.D.\ thesis of David Jack \cite{jack}:
and the second directional diffusion model is
\begin{align*}
D_r &= C_1 \gamma^{-1} \int_{\brho \in S^2} |\brho\cdot(\br\times(\dot{\tilde\brho}-\dot{\tilde \br}))|^2 \psi(\brho) \, d\brho +
C_2 \gamma \\
&= C_1 \gamma^{-1} \int_{\brho \in S^2} \bigl(|\brho\cdot(\br\times\dot{\tilde\brho})|^2 + |\brho\cdot(\br\times\dot{\tilde\br})|^2\bigr) \psi(\brho) \, d\brho +
C_2 \gamma .
\end{align*}
\begin{itemize}
\item The parameter set as {\tt do\_dd=1} will cause the directional diffusion model to be used.
\item The parameter {\tt lambda} is the Jeffery's parameter $\lambda$.
\item The parameters {\tt C1} and {\tt C2} are the directional diffusion parameters $C_1$ and $C_2$.
\end{itemize}

\section*{The RSC model of Wang, O'Gara and Tucker}

\noindent
This causes an artificial scaled reduction of the evolution of $\psi$, the ``Reduced-Strain Closure" model.  This can be used in conjunction with any of the other models.  It is described in \cite{rsc}, and the file {\tt rsc.pdf}.  It causes the P.D.E.\ 
$$ \frac{\partial}{\partial t}\psi(\br) = F(\br),$$
to be replaced by
$$ \frac{\partial}{\partial t}\psi(\br) = F(\br) - \tfrac{15}{8\pi}(1-\kappa)(\br\br-\tfrac15 I):\mathbb M:\int_{S^2} F(\brho) \brho\brho \, d\brho,$$
where $\mathbb M = \sum_{i=1}^3 \e_i\e_i\e_i\e_i$, with $\e_1$, $\e_2$, $\e_3$ being the orthonormal eigenvectors of $A$.

This method has intellectual property restrictions \cite{rsc-pat}.  If you use {\tt do\_rsc=1}, the program will interrogate you to see if you comply with this patent.  You may switch off the interrogation by setting the environment variable {\tt MAY\_USE\_PATENT\_7266469}.

\begin{itemize}
\item The parameter set as {\tt do\_rsc=1} causes RSC to take place.
\item The parameter {\tt kappa} sets the scale reduction factor $\kappa$.
\end{itemize}

\section*{The ARD model of Phelps and Tucker}

\noindent
This ``Anisotropic Rotary Diffusion'' model is described in \cite{ard}.  This is the same form as the Koch model, with
$$ D_r = b_1 \gamma I + b_2 \gamma A + b_3 \gamma A^2 + \tfrac12{b_4} \Gamma + \tfrac14{b_5}\gamma^{-1} \Gamma^2 .$$

\begin{itemize}
\item The parameter set as {\tt do\_ard=1} causes ARD to be used.
\item The parameters {\tt b1}, {\tt b2}, {\tt b3}, {\tt b4} and {\tt b5} set $b_1$, $b_2$, $b_3$, $b_4$ and $b_5$.
\end{itemize}

\section*{Variable diffusion and variable lambda models}

\noindent
Define:
$$ s = \gamma^{-1} \int_{S^2} \Gamma:\brho\brho \psi(\brho) \, d\brho = \gamma^{-1} A :\Gamma .$$
Note that it can be shown that $|s| \le \sqrt2$.

The variable diffusion model sets $D_r = \gamma f(s)$, and the variable lambda model sets $\lambda = g(s)$, for functions $f(s)$ and $g(s)$.
\begin{itemize}
\item The parameter set as {\tt do\_vd}=1 causes variable diffusion to occur.
\item The parameter set as {\tt do\_vl}=1 causes variable lambda to occur.
\item In the case {\tt do\_vd}=1, the function $f(s)$ is defined using the parameter {\tt vd\_fun}.
\item In the case {\tt do\_vl}=1, the function $g(s)$ is defined using the parameter {\tt vl\_fun}.  In this case the parameter {\tt lambda} should NOT be set.
\end{itemize}
Functions are expressed using formulae involving `{\tt +}', `{\tt -}', `{\tt *}', `{\tt /}', `{\tt \verb|^|}' (exponentiation), `{\tt m}' (minimum), `{\tt M}' (maximum), parentheses `{\tt(}' and `{\tt)}', and the variable `{\tt s}'.  So, for example, `{\tt 1-5*(s M(-s)) M 0.2}' denotes $\max\{1-5|s|,0.2\}$.

\section*{Acknowledgments}

\noindent
This work was supported by the National Science Foundation, Division of Civil, Mechanical and Manufacturing Innovation, award number 0727399 (NSF CMMI 0727399).

\begin{thebibliography}{99}
\bibitem{jeffery}
G.B. Jeffery, The Motion of Ellipsoidal Particles Immersed in a Viscous Fluid, Proceedings of the Royal Society of London A {\bf 102}, 161-179, (1923).
\bibitem{folgar}
F.P. Folgar and C.L. Tucker, Orientation Behavior of Fibers in Concentrated Suspensions, Jn. of Reinforced Plastics and Composites {\bf 3}, 98-119 (1984).
\bibitem{koch}
D.L. Koch, A Model for Orientational Diffusion in Fiber Suspensions, Physics of Fluids {\bf 7}, 2086-2088 (1995).
\bibitem{jack}
D.A. Jack, Advanced analysis of short-fiber polymer composite material behavior, Ph.D.\ thesis, University of Missouri (2006).
\bibitem{rsc-pat}
C.L. Tucker III, J. Wang, and J.F. O'Gara, Method and article of manufacture for determining a rate of change of orientation of a plurality of fibers disposed in a fluid. U.S. Patent No. 7,266,469 (2007).
\bibitem{rsc}
Jin Wang, John F. O'Gara, and Charles L. Tucker, III, An objective model for slow orientation kinetics in concentrated fiber suspensions: Theory and rheological evidence, J. Rheology {\bf 52}, 1179-1200 (2008).
\bibitem{ard}
Jay H. Phelps and Charles L. Tucker III, An anisotropic rotary diffusion model for fiber orientation in short- and long-fiber thermoplastics, Journal of Non-Newtonian Fluid Mechanics, to appear (2008).
\end{thebibliography}


\end{document}
