\documentclass{amsart}

\usepackage{psfrag}
\usepackage{graphicx}
\usepackage{url}
\usepackage{amsmath,amsfonts}
%\usepackage{fullpage}

\newcommand\br{{\mathbf r}}
\newcommand\w{{\mathbf w}}
\newcommand\bL{{\mathbf L}}
\newcommand\boldf{{\mathbf f}}
\newcommand\bz{{\mathbf z}}
\newcommand\bnabla{{\boldsymbol \nabla}}
\newcommand\e{{\mathbf e}}
\newcommand\q{{\mathbf q}}

\begin{document}

\section{Differential equations on the unit sphere}

The theoretical underpinnings of this package are described in \cite{montgomerysmith:10}.

The objective of the software is to provide a systematic spherical harmonic method to numerically solve partial differential equations on the two
dimensional sphere, $S = \{\br = (x,y,z):|\br| = x^2+y^2+z^2 = 1\}$, or in spherical coordinates, $\br = (x,y,z) = (\sin\theta
\cos\phi,\sin\theta\sin\phi,\cos\theta)$, where $0\le\theta\le\pi$, $0\le\phi\le2\pi$.  The partial differential equations considered here are of the
form
\begin{equation}
\label{pde}
\frac\partial{\partial t}\psi = F(\br,\bnabla) \psi,
\end{equation}
where $\psi$ is a function (for example, the probability distribution function) defined on the sphere. Here $\bnabla$ is the gradient operator
restricted to the sphere that is defined as
\begin{equation}
\begin{split}
\bnabla &= (\nabla_x,\nabla_y,\nabla_z) \\
&= \left(\cos\theta\cos\phi\frac{\partial}{\partial\theta} - \csc\theta\sin\phi \frac{\partial}{\partial\phi},\cos\theta\sin\phi\frac{\partial}{\partial\theta} + \csc\theta\cos\phi \frac{\partial}{\partial\phi},-\sin\theta\frac{\partial}{\partial\theta} \right) \\
&= \left((1-x^2)\frac\partial{\partial x}-xy\frac\partial{\partial y}-xz\frac\partial{\partial z},
         -xy\frac\partial{\partial x}+(1-y^2)\frac\partial{\partial y}-yz\frac\partial{\partial z},\right.\\
&\qquad\qquad\left.
         -xz\frac\partial{\partial x}-yz\frac\partial{\partial y}+(1-z^2)\frac\partial{\partial
         z}\right) .
\end{split}
\end{equation}
In the above, $F$ is a polynomial in six variables, with the proviso that it matters in which order the terms of each monomial part are written. We
believe that this includes many of the published, if not all, of the partial differential equations that describe the evolution of
the orientation distribution function for short fiber suspensions.

The spherical harmonic approach converts equation~\eqref{pde} to a
system of ordinary differential equations written as
\begin{equation}
\label{spherical-pde}
\frac\partial{\partial t} \hat\psi_l^m = \sum_{l'=0}^\infty \sum_{m'=-l'}^{l'} c_{l,l'}^{m,m'} \hat\psi_{l'}^{m'} ,
\end{equation}
where $\hat\psi_l^m$ ($l \ge 0$, $|m| \le l$) are the spherical harmonic coefficients of $\psi$, defined in
equation~\eqref{spherical-harmonic-coefficient} below, and $c_{l,l'}^{m,m'}$ are the coefficients defined through equation~\eqref{decompose F}.

The software provides a systematic algorithm for calculating the coefficients $c_{l,l'}^{m,m'}$ appearing in equation~\eqref{spherical-pde} (and also in equation~\eqref{decompose F} below). This algorithm makes it possible to readily solve all differential
equations on the unit sphere of the form shown in equation~\eqref{pde}.

\section{Description of the model equations}

To illustrate the software, we consider variations of Jeffery's equation \cite{jeffery:23} which describes the motion of fibers in a
moving fluid with vorticity $\w$ and rate of deformation tensor $\Gamma$. Jeffery's equation is
often written in terms of the fiber orientation distribution function $\psi$ and the fiber aspect ratio parameter $-1 \le \lambda\le1$ as
\begin{equation}
\label{jeffery's}
\frac\partial{\partial t}\psi = J \psi := -
\tfrac12\bnabla\cdot(\w \times \br \psi + \lambda(\Gamma\cdot\br -
\Gamma:\br\br\br) \psi) ,
\end{equation}
where we note that the right hand side is in the form of equation~\eqref{pde}. A variation of equation~\eqref{jeffery's} is Jeffery's equation with
rotary diffusion as expressed by Bird \cite{bird:87b} as
\begin{equation}
\label{jeffery's-folgar-tucker}
\frac\partial{\partial t}\psi = J \psi + \bnabla\cdot\bnabla(D_r
\psi) ,
\end{equation}
where $D_r$ captures the effect of fiber interaction and depends upon the flow kinetics. Folgar and Tucker \cite{folgar:84} selected $D_r = C_I \dot{\gamma}$ where $\dot{\gamma} =
\left(\frac12\Gamma:\Gamma\right)^{1/2}$ and $C_I$ is a constant that depends upon the volume fraction and aspect ratio of the
fibers.

Another example is anisotropic diffusion such as that proposed by Koch \cite{koch:95} in the
differential equation
\begin{equation}
\label{koch} \frac\partial{\partial t}\psi = J \psi +
\bnabla\cdot(I-\br\br)\cdot D_r\cdot\bnabla\psi = J \psi +
(\bnabla-2\br)\cdot D_r\cdot\bnabla\psi
\end{equation}
where the anisotropic diffusion matrix $D_r$ is defined in terms of the model parameters $C_1$ and $C_2$ (see Koch \cite{koch:95} for more detail) as
\begin{equation}
\label{kochDr}
D_r = C_1 {\dot\gamma}^{-1} (\Gamma:\mathbb A:\Gamma)I + C_2
{\dot\gamma}^{-1} \Gamma:\mathcal A:\Gamma .
\end{equation}
In the above, the $(I-\br\br)$ term serves to project vectors onto the surface of the sphere. We note that this term is not explicitly included in
Koch's formula in \cite{koch:95}, however it's existence is implied in the paragraph following the original introduction.

The anisotropic diffusion matrix $D_r$ defined in equation~\eqref{kochDr} is written in terms of the 2nd, 4th and 6th order moment tensors of $\psi$
which are, respectively, defined as
\begin{equation}
\label{momenttensors}
 A := \int_S \psi \br\br \, d\br ,\qquad
\mathbb A := \int_S \psi \br\br\br\br \, d\br ,\qquad \text{and}
\qquad \mathcal A := \int_S \psi \br\br\br\br\br\br \, d\br .
\end{equation}
In these integrals, we adopt the common definition for the integral of a function $F(\br)$ over the surface of the unit sphere as
\begin{equation}
\int_S F(\br) \, dr := \int_{\theta=0}^\pi \int_{\phi=0}^{2\pi}
F(\br) \sin\theta\,d\phi\,d\theta ,
\end{equation}
and we note here that the moment tensors $A$, $\mathbb A$, and $\mathcal{A}$ can be expressed in terms of spherical harmonics as shown below in
equation~\eqref{moment tensor}.

Care must be exercised in not confusing the differential geometry on the surface of the sphere $S$ with the differential geometry
in the three dimensional space $\mathbb R^3$ it is embedded in.  Thus, for example, integration by parts
is the slightly unexpected form
\begin{equation}
\label{by-parts}
\int_S \boldf \cdot \bnabla g\, d\br  = - \int_S (\bnabla\cdot (I-\br\br)\cdot\boldf) g\, d\br = \int_S ((2\br - \bnabla)\cdot\boldf)g \, d\br ,
\end{equation}
which reduces to the usual integration by parts when $\boldf$ is
tangential to the surface of the sphere. We also recall the so
called angular momentum operator
\begin{equation}
\label{ang-momentum-ops}
\bL = -i\br\times\bnabla = (L_x,L_y,L_z) =
-i\left(y\frac\partial{\partial z} - z\frac\partial{\partial y} ,
z\frac\partial{\partial x} - x\frac\partial{\partial z} ,
x\frac\partial{\partial y} - y\frac\partial{\partial x} \right) ,
\end{equation}
where, as usual,  $i$ denotes the complex number satisfying $i^2=-1$.  Integration by parts for the angular momentum operator is more straightforward, that is,
\begin{equation}
\label{by-parts-l}
\int_S \boldf \cdot \bL g\, d\br  = -\int_S (\bL\cdot\boldf) g\, d\br .
\end{equation}
Note that $\bL\times\bL = \bnabla\times\bnabla = i\bL$, and in particular $\nabla_x$, $\nabla_y$ and $\nabla_z$
do not commute with each other. We also use the formulae $\br\cdot\bnabla f=\br\cdot\bL f=\bL\cdot(\br f) = 0$ and $\bnabla\cdot(\br f) = 2f$.

\section{Spherical harmonics solutions}

Any square integrable function $\psi$ defined on the unit sphere may be written as a Fourier series like representation in terms of the spherical
harmonics $Y_l^m(\theta,\phi)$ for $l \ge 0$ and $|m| \le l$ as
\begin{equation}
\label{spherical-series}
\psi = \sum_{l=0}^\infty \sum_{m=-l}^l
\hat\psi_l^m Y_l^m ,
\end{equation}
where $Y_l^m$ are the spherical harmonic functions as defined in \cite{weisstein:06a}.
The coefficients $\hat\psi_l^m$ in equation~\eqref{spherical-series}
are evaluated from
\begin{equation}
\label{spherical-harmonic-coefficient}
\hat\psi_l^m = \int_S \psi
\bar Y_l^m d\br .
\end{equation}

To obtain the system of ordinary differential equations~\eqref{spherical-pde}, we integrate equation~\eqref{pde} against $\bar Y_l^m$ over the unit
sphere to obtain its adjoint or weak form as
\begin{equation}
\label{weakform}
\frac\partial{\partial t} \int_S \psi \bar Y_l^m
\,d\br = \int_S \psi F^*(\br,\bnabla) \bar Y_l^m \,d\br ,
\end{equation}
where the right hand side follows from integration by parts and the term $F^*$ denotes any polynomial $F$ in which each monomial term is written in
reverse order with the substitution of $2\br - \bnabla$ for $\bnabla$. Therefore, applying equation~\eqref{weakform}, Jeffery's
equation~\eqref{jeffery's} is defined by
\begin{equation}
J^* = \tfrac12(\w\cdot(\br\times\bnabla) + \lambda\br\cdot\Gamma\cdot\bnabla),
\end{equation}
Similarly, the two extensions of Jeffery's equation that include diffusion appearing in equations~\eqref{jeffery's-folgar-tucker} and~\eqref{koch}
(expressed as $\frac\partial{\partial t}\psi = F \psi$), respectively become
\begin{equation}
\label{FStarFT}
F^* = J^* + D_r \bnabla\cdot\bnabla ,
\end{equation}
and
\begin{equation}
F^* = J^* + (\bnabla-2\br)\cdot D_r\cdot\bnabla .
\end{equation}
In all cases, we can decompose
\begin{equation}
\label{decompose F}
F^*(\br,\bnabla) \bar Y_l^m = \sum_{l'=0}^\infty \sum_{m'=-l'}^{l'} c_{l,l'}^{m,m'} \bar Y_{l'}^{m'} ,
\end{equation}
and therefore obtain the spherical harmonic representation shown in equation~\eqref{spherical-pde} above.

\section{The ``spherical'' program algorithm}

An automated algorithm is implemented to compute the coefficients $c_{l,l'}^{m,m'}$ in equation~\eqref{decompose F} for
any differential equation satisfying the criteria given above (and, therefore, the $\mathcal C_{l,m}$ in equation~\eqref{moment tensor}). The
algorithm is to recursively apply the replacement rules described in equation~\eqref{rules} until no further substitutions can be made. Here ``op''
denotes any of $z$, $L_z$, $L_+$, or $L_-$ operations, and $\mathcal Y$ denotes any linear combination of the $\bar Y_l^m$'s.
\begin{equation}
\label{rules}
\begin{split}
\text{op}(\mathcal Y \pm c \bar Y_l^m) &\to \text{op}(\mathcal Y) \pm c \, \text{op}(\bar Y_l^m), \\
x \mathcal Y &\to  z i L_y(\mathcal Y) - i L_y(z\mathcal Y), \\
y \mathcal Y &\to  i L_x(z\mathcal Y) - z i L_x(\mathcal Y), \\
z\bar Y_l^m &\to \sqrt{\frac{(l+m)(l-m)}{(2l-1)(2l+1)}} \bar Y_{l-1}^m + \sqrt{\frac{(l+m+1)(l-m+1)}{(2l+1)(2l+3)}} \bar Y_{l+1}^m ,\\
\nabla_x \mathcal Y &\to z i L_y(\mathcal Y) - y i L_z(\mathcal Y),  \\
\nabla_y \mathcal Y &\to x i L_z(\mathcal Y) - z i L_x(\mathcal Y),  \\
\nabla_z \mathcal Y &\to y i L_x(\mathcal Y) - x i L_y(\mathcal Y),  \\
L_x (\mathcal Y) &\to \tfrac 12 \left(L_+ (\mathcal Y) + L_- (\mathcal Y)\right), \\
L_y (\mathcal Y) &\to -\tfrac i2 \left(L_+ (\mathcal Y) - L_- (\mathcal Y)\right), \\
L_z \bar Y_l^m &\to -m\bar Y_l^m ,\\
L_+ \bar Y_l^m &\to - \sqrt{(l+m)(l-m+1)}\bar Y_l^{m-1}, \\
L_- \bar Y_l^m &\to - \sqrt{(l-m)(l+m+1)}\bar Y_l^{m+1}
\end{split}
\end{equation}
It is important to note that $x \bar Y_l^m$, $y \bar Y_l^m$, $z \bar Y_l^m$, $\nabla_x\bar Y_l^m$, $\nabla_y \bar Y_l^m$, and $\nabla_z \bar Y_l^m$
involve $\bar Y_{l'}^{m'}$ for $l'$ and $m'$ that differ from $l$ and $m$, respectively, by at most one.  The script described in the appendix uses
the recursive algorithm in equation~\eqref{rules} to create threaded functions in the programming language
C which in turn are used in an iterative procedure for computing the solution to the differential equation in equation~\eqref{pde}.

Moment tensors may be computed using this algorithm as well. For example, to compute the 6th order moment tensor $\mathcal A$ in
equation~\eqref{momenttensors} we simply expand
\begin{equation}
\label{expand Y_0^0} \br\br\br\br\br\br \bar Y_0^0 = \sum_{l=0}^6
\sum_{m=-l}^l \mathcal C_{l,m} \bar Y_l^m,
\end{equation}
where $\mathcal C_{l,m}$ is the tensor of rank six composed of coefficients calculated by applying $x$, $y$ and $z$ to $\bar Y_0^0$ six times using the algorithm. It follows that since $\bar Y_0^0 = 1/\sqrt{4\pi}$, the 6th order orientation tensor
becomes
\begin{equation}
\label{moment tensor} \mathcal A = \sqrt{4\pi} \int_S \psi
\br\br\br\br\br\br \bar Y_0^0 \, d\br = \sqrt{4\pi} \sum_{l=0}^6
\sum_{m=-l}^l \mathcal C_{l,m} \hat\psi_l^m ,
\end{equation}
where the orthogonality of the spherical harmonics are used to simplify the final result.

\section{The ``spherical'' software package}

The program developed to implement our spherical harmonics
simulation procedure described above which computes the coefficients
in equation~\eqref{spherical-pde}, and then convert these terms into
a C program, may be found at
\url{https://github.com/stephenmontgomerysmith/spherical}. The
program is written in \emph{perl}, and makes use of either the
commercial computer algebra system \emph{Mathematica} by Wolfram
Research, or the open-source computer algebra system \emph{Maxima}.
The scripts are designed to work in a Unix like environment.

The program that evaluates spherical harmonics coefficients is a
\emph{perl} script which applies the rules in~\eqref{rules}. It is
invoked with the command sequence ``{\tt perl expand-method.pl}''.
The user may type in an expression involving {\tt x}, {\tt y}, {\tt
z}, {\tt dx}, {\tt dy}, {\tt dz}, {\tt lx}, {\tt ly}, or {\tt lz}
representing ``multiply by $x$,'' ``multiply by $y$,'' ``multiply by
$z$,'' $\nabla_x$, $\nabla_y$, $\nabla_z$, $L_x$, $L_y$, or $L_z$
respectively, or any number, including {\tt I} for $i$, and any
appropriate combination of parentheses, addition and subtraction,
and ``{\tt *}'' which represents composition of the operators.  It
outputs a sequence of lines, which when added together, represents
the effect of the expression upon $\bar Y_l^m$.  Each line has three
entries separated by semicolons, $l'$;$m'$;$c$, and represents
$c\bar Y_{l+l'}^{m+m'}$.  The third entry is written in the
programming language C, using complex numbers as described in
\cite{c:99}.  So, for example, the input
\begin{verbatim}
%perl expand-method.pl
> -2*I*(dy-y)
\end{verbatim}
yields the output
\begin{verbatim}
-1;-1;((l*sqrt(-1+l+m)*sqrt(l+m))/(sqrt(-1+2*l)*sqrt(1+2*l)))+(0)*I
-1;1;((l*sqrt(-1+l-m)*sqrt(l-m))/(sqrt(-1+2*l)*sqrt(1+2*l)))+(0)*I
1;-1;(((1+l)*sqrt(1+l-m)*sqrt(2+l-m))/(sqrt(1+2*l)*sqrt(3+2*l)))+(0)*I
1;1;(((1+l)*sqrt(1+l+m)*sqrt(2+l+m))/(sqrt(1+2*l)*sqrt(3+2*l)))+(0)*I
\end{verbatim}
This result reflects the equation
\begin{equation*}
\begin{split}
-2i(\nabla_y - y)\bar Y_l^m = &
\tfrac{l\sqrt{(l+m-1)(l+m)}}{\sqrt{(2l-1)(2l+1)}} \bar Y_{l-1}^{m-1}
+ \tfrac{l\sqrt{(l-m-1)(l-m)}}{\sqrt{(2l-1)(2l+1)}} \bar Y_{l-1}^{m+1} \\
& + \tfrac{(l+1)\sqrt{(l-m+1)(l-m+2)}}{\sqrt{(2l+1)(2l+3)}} \bar Y_{l+1}^{m-1}
+ \tfrac{(l+1)\sqrt{(l+m+1)(l+m+2)}}{\sqrt{(2l+1)(2l+3)}} \bar Y_{l+1}^{m+1} .
\end{split}
\end{equation*}

The program ``{\tt perl expand-method.pl}'' is called by another
script invoked as ``{\tt perl expand-iterate.pl <filename>}'' which
converts a configuration file into a C program to numerically solve
the spherical harmonics differential equation. For example,
Jeffery's equation (written here in weak form) with Koch diffusion
can be written as below
\begin{verbatim}
void compute_psidot(complex* psidot, complex* psi,
                    double lambda, double w[3], double g[3][3]) {
  double a4[3][3][3][3], a6[3][3][3][3][3][3], Dr[3][3];

  tensor4(psi,a4); tensor6(psi,a6);
  ... compute Dr from g, a4 and a6 ...

  @spherical_iterate {
    psidot[@index] =
      /* Jeffery's equation */
        w[0]*@method(psi,0.5*y*dz-0.5*z*dy)
      + w[1]*@method(psi,0.5*z*dx-0.5*x*dz)
      + w[2]*@method(psi,0.5*x*dy-0.5*y*dx)
      + lambda*g[0][0]*@method(psi,0.5*x*dx)
      + lambda*g[0][1]*@method(psi,0.5*x*dy+0.5*y*dx)
      + lambda*g[0][2]*@method(psi,0.5*x*dz+0.5*z*dx)
      + lambda*g[1][1]*@method(psi,0.5*y*dy)
      + lambda*g[1][2]*@method(psi,0.5*y*dz+0.5*z*dy)
      + lambda*g[2][2]*@method(psi,0.5*z*dz)
      /* Koch diffusion */
      + Dr[0][0]*@method(psi,(dx-2*x)*dx)
      + Dr[0][1]*@method(psi,(dx-2*x)*dy+(dy-2*y)*dx)
      + Dr[0][2]*@method(psi,(dx-2*x)*dz+(dz-2*z)*dx)
      + Dr[1][1]*@method(psi,(dy-2*y)*dy)
      + Dr[1][2]*@method(psi,(dy-2*y)*dz+(dz-2*z)*dy)
      + Dr[2][2]*@method(psi,(dz-2*z)*dz);
  }
}
\end{verbatim}

To calculate the coefficients for the moment tensors, as in
equation~\eqref{expand Y_0^0}, if the environmental variable {\tt
TENSOR} is set to a non-zero value, then {\tt expand-method.pl}
applies the input expression to $\bar Y_0^0$, and outputs the
answers in floating point format.  So, for example the input
\begin{verbatim}
%env TENSOR=1 perl expand-method.pl
> x*y*z*z
\end{verbatim}
yields
\begin{verbatim}
2;-2;(0)+(-0.026082026547865053022)*I
2;2;(0)+(0.026082026547865053022)*I
4;-2;(0)+(-0.030116930096841707924)*I
4;2;(0)+(0.030116930096841707924)*I
\end{verbatim}
This results reflects the 1233 component of 4th order moment tensor
\begin{equation*}
\mathbb A_{1233} =
i\sqrt{4\pi}\left(\tfrac{\sqrt{30}}{210}(\hat\psi_2^2-\hat\psi_2^{-2})
+\tfrac{\sqrt{10}}{105}(\hat\psi_4^2-\hat\psi_4^{-2})\right).
\end{equation*}
This is used by ``{\tt perl make-tensor.pl $n$}'' and ``{\tt perl
make-reverse-tensor.pl $n$}'', which create C programs to convert,
respectively, spherical harmonic coefficients to moment tensors of
rank $n$, and vice-versa.

\bibliographystyle{plain}
\bibliography{info_08}

\end{document}
